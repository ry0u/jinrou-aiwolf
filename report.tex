\documentclass[a4paper]{jarticle}

\usepackage{url}
\usepackage{here}
\usepackage{listings}
\usepackage{wrapfig}
\usepackage{ascmac}
\usepackage[dvipdfmx]{graphicx}
\usepackage[top=30truemm,bottom=30truemm,left=25truemm,right=25truemm]{geometry}

\begin{document}

自分のプログラムが以下のことを実現できているのか,確信が全く無いのでそれっぽい動きはするかもしれないが,しないかもしれない.

\section{村人陣営}

\subsection{占い}

\begin{itemize}
    \item 初日の朝最速でCO
    \item その後占い結果も朝最速で伝える
    \item 占い先は1COなら自由でよい
    \item 2COの時は対抗の白出しを占うことえ確定白を作る(Randomで分ける)
\end{itemize}

\subsection{霊能}

\begin{itemize}
    \item 基本的に余りCOしない.自分に投票が集まりそうならCOもあり
    \item COするタイミングは偽COに対するCCOや占いが黒だしした先がつられた時など
    \item CCOが成功した場合ローラーするのがよい
\end{itemize}

\subsection{狩人}

\begin{itemize}
    \item ひたすら占いを守るのが仕事
    \item たまに色気を出して1COの霊能や,確定白を守る
    \item 狼から見て真占いが確定する状況では,狩人からは分からなくても,占いを優先して守りたい
    \item 奇数から偶数は釣り数は増えないが,偶数から奇数は増えるので,偶数の時には強気に読みに行く
\end{itemize}

\subsection{村人}

\begin{itemize}
    \item COは基本しない
    \item 投票先はグレランか確定白に合わせるかローラーの三択
    \item ローラーに完全にやり切るのが大事
    \item 占い確定時に村人COをして釣りで狩人を守るという選択肢も無くな無い
\end{itemize}

\section{狼陣営}

\subsection{狼}

\begin{itemize}
    \item 占いが確定し,狩人で守られ続けると割とどうしようもないので,狂人が占いCOしない場合は自分が占いCO
    \item 投票が集まりそうな場合は霊能COをすることで吊りに2手かけさせる(これは一度のみ)
    \item 占いに黒だしされた場合は何を行っても吊られるので諦める
    \item 噛み対象は,村から見て確定白や霊能CO等を噛み,狩人が吊られたり,噛めたりしそうなら占いを噛む
    \item 一度なら守られても吊りは増えないので,強気に占いを噛みにいく
    \item 狂人占い師の黒だしや狼への白だしによって真占いが見えた場合,すぐに噛む
    \item 狼の吊り投票合わせは強力だが,下手すると見破られて狼が全滅するのでしない.
\end{itemize}

\subsection{狂人}
\begin{itemize}
    \item 占いCOして場にも確定的な情報を減らすのが仕事
    \item 何もしないで潜伏しているので,狼がリスクを追わないように,COが出たらCCOする.
    \item COした役職を演じきる
\end{itemize}

\end{document}
